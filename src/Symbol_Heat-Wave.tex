\documentclass[a4paper,papersize,10pt,landscape]{jsarticle}


%ページ設定・レイアウト設定
%1 pt = 1/72 inch
%1 inch = 25.4 mm

\setlength{\oddsidemargin}{-30pt}
%\setlength{\evensidemargin}{-28.8pt}

\setlength{\textwidth}{270mm}
%\setlength{\textwidth}{\paperwidth}
%\addtolength{\textwidth}{-70pt}

\setlength{\topmargin}{-55pt}
\setlength{\headheight}{17pt}
\setlength{\headsep}{8pt}
\setlength{\footskip}{23pt}
\setlength{\textheight}{\paperheight}
\addtolength{\textheight}{-82pt}%731pt-42*2


%読み込むパッケージ
\usepackage{lastpage}% 全ページ数を出力するため
\usepackage{fancyhdr}% ヘッダー・フッターを簡便に作成
\usepackage[dvips]{graphicx}
\usepackage[T1]{fontenc}
\usepackage{textcomp}


%ヘッダとフッタ
\pagestyle{fancy}
\renewcommand{\headrulewidth}{0pt}
\rfoot{\jobname{}.tex (2017-02-10)}
\rhead[RO]{\thepage{}/{}\pageref{LastPage}}
\cfoot{}



%表の設定
\def\row#1#2#3#4#5#6#7{#1 &\hfil #2 &\hfil #3 &\hfil #4 &\hfil #5 &\hfil #6 & #7 \\ }
\def\ppppppp{p{70mm}p{30mm}p{40mm}p{15mm}p{20mm}p{50mm}p{15mm}}
\renewcommand{\arraystretch}{1.2}

%太罫線の定義
\makeatletter
\def\Hline{%
\noalign{\ifnum0=`}\fi\hrule \@height 2pt \futurelet
\reserved@a\@xhline}
\makeatother




\begin{document}

\textbf{\Large 量記号と単位 -- 『熱・波動』(大日本図書)}
\vskip1zh

\textbf{基本的な物理量}\\
\begin{tabular}{\ppppppp}\Hline
  \hfil 意味	&\hfil 科学量(日本語)	&\hfil 科学量(英語)&\hfil 量記号	&\hfil 単位記号	&\hfil 単位(英語)	&\hfil 教科書 \\\hline
  \row{}{長さ}{length}{$L, l$}{m}{meter}{}
  \row{}{半径}{radius}{$R, r$}{m}{meter}{}
  \row{}{高さ}{height}{$H, h$}{m}{meter}{}
  \row{}{体積}{volume}{$V$}{m$^3$}{cubic meter}{}
  \row{}{面積}{area}{$S$}{m$^2$}{square meter}{}
  \row{}{質量}{mass}{$M, m$}{kg}{kilogram}{}
  \row{}{時間}{time}{$t$}{s}{second}{}
  \row{}{速度}{velocity}{$v$}{m/s}{meter per second}{}
  \row{}{温度}{temperature}{$T, t$}{K}{kelvin}{}
\Hline
\end{tabular}
\vfil



\textbf{熱編}\\
\begin{tabular}{\ppppppp}\Hline
  \hfil 意味	&\hfil 科学量(日本語)	&\hfil 科学量(英語)&\hfil 量記号	&\hfil 単位記号	&\hfil 単位(英語)	&\hfil 教科書 \\\hline
  \row{温冷の度合いを表す指標}{温度}{temperature}{$T, t$}{K}{kelvin}{p.~11}
  \row{物体の温まりにくさ}{熱容量}{heat capacity}{$C$}{J/K}{joule per kelvin}{p.~15}
  \row{単位質量あたりの物質の温まりにくさ}{比熱}{specific heat}{$c$}{J/g$\cdot$K}{joule per gram kelvin}{p.~15}
  \row{ものを温めるエネルギー}{熱量}{quantity of heat}{$Q, q$}{J}{joule}{p.~15}
  \row{}{長さ}{length}{$l$}{m}{meter}{p.~22}
  \row{ものを温めた時の伸び具合}{線膨張率}{略}{$\alpha$}{1/K}{per kelvin}{p.~22}
  \row{}{体積}{volume}{$V$}{m$^3$}{cubic meter}{p.~22}
  \row{ものを温めた時の膨らみ具合}{体膨張率}{略}{$\alpha$}{1/K}{per kelvin}{p.~22}
  \row{}{面積}{area}{$S$}{m$^2$}{square meter}{p.~26}
  \row{熱の伝えやすさ}{熱伝導率}{thermal conductivity}{$\kappa$}{W/m$\cdot$K}{watt per meter kelvin}{p.~26}
  \row{単位面積当たりに作用する力}{圧力}{pressure}{$p$}{Pa}{pascal}{p.~30}
  \row{不可逆性を表す量}{エントロピー}{entropy}{$S$}{J/K}{joule per kelvin}{p.~76}
\Hline
\end{tabular}

\newpage
\textbf{波動編}\\
\begin{tabular}{\ppppppp}\Hline
  \hfil 意味	&\hfil 科学量(日本語)	&\hfil 科学量(英語)&\hfil 量記号	&\hfil 単位記号	&\hfil 単位(英語)	&\hfil 教科書 \\\hline
 \row{変位の最大値}{振幅}{amplitude}{$A$}{m}{meter}{p.~106}
 \row{隣り合う山の間の長さ}{波長}{wave length}{$\lambda$}{m}{meter}{p.~106}
 \row{}{時間}{time}{$t$}{s}{second}{p.~106}
 \row{}{速度}{velocity}{$v$}{m/s}{meter per second}{p.~106}
 \row{媒質が一回振動する時間}{周期}{period}{$T$}{s}{second}{p.~107}
 \row{媒質が1秒間に振動する回数}{振動数}{frequency}{$f$}{Hz}{hertz}{p.~107}
 \row{波が違う種類の媒質に進む時の曲がり具合}{屈折率}{refraction index}{$n$}{---}{---}{p.~127}
 \row{}{張力}{tension}{$S$}{N}{newton}{p.~139}
 \row{}{線密度}{liear density}{$\rho$}{kg/m}{kilogram per meter}{p.~139}
 \row{鏡・レンズから焦点までの距離}{焦点距離}{focal length}{$f$}{m}{meter}{p.~163}
 \row{像と物体の長さの比}{倍率}{magnification}{$m$}{---}{---}{p.~164}
\Hline
\end{tabular}
\vskip2zh


*英語の綴りは、電子辞書で発音も確認してください.

*量記号は斜体(イタリック体)で書きます.

*単位記号は立体(ローマン体)で書きます.

*人命由来の単位記号は大文字で書き始めます.しかし、綴りの時には小文字で書きます.



\end{document}
