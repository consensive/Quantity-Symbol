\documentclass[a4paper,papersize,10pt,landscape]{jsarticle}


%ページ設定・レイアウト設定
%1 pt = 1/72 inch
%1 inch = 25.4 mm

\setlength{\oddsidemargin}{-30pt}
%\setlength{\evensidemargin}{-28.8pt}

\setlength{\textwidth}{270mm}
%\setlength{\textwidth}{\paperwidth}
%\addtolength{\textwidth}{-70pt}

\setlength{\topmargin}{-55pt}
\setlength{\headheight}{17pt}
\setlength{\headsep}{8pt}
\setlength{\footskip}{23pt}
\setlength{\textheight}{\paperheight}
\addtolength{\textheight}{-82pt}%731pt-42*2


%読み込むパッケージ
\usepackage{lastpage}% 全ページ数を出力するため
\usepackage{fancyhdr}% ヘッダー・フッターを簡便に作成
\usepackage[dvips]{graphicx}
\usepackage[T1]{fontenc}
\usepackage{textcomp}


%ヘッダとフッタ
\pagestyle{fancy}
\renewcommand{\headrulewidth}{0pt}
\rfoot{\jobname{}.tex (2017-02-10)}
\rhead[RO]{\thepage{}/{}\pageref{LastPage}}
\cfoot{}



%表の設定
\def\row#1#2#3#4#5#6#7{#1 &\hfil #2 &\hfil #3 &\hfil #4 &\hfil #5 &\hfil #6 & #7 \\ }
\def\ppppppp{p{70mm}p{30mm}p{40mm}p{15mm}p{20mm}p{50mm}p{15mm}}
\renewcommand{\arraystretch}{1.2}

%太罫線の定義
\makeatletter
\def\Hline{%
\noalign{\ifnum0=`}\fi\hrule \@height 2pt \futurelet
\reserved@a\@xhline}
\makeatother




\begin{document}

\textbf{\Large 量記号と単位 -- 『電磁気・原子』(大日本図書)}
\vskip1zh


\textbf{第1章 物体の運動}\\
\begin{tabular}{\ppppppp}\Hline
  \hfil 意味	&\hfil 科学量(日本語)	&\hfil 科学量(英語)&\hfil 量記号	&\hfil 単位記号	&\hfil 単位(英語)	&\hfil 教科書 \\\hline
  \row{物体の2点間の距離}{距離}{distance}{$x$}{m}{meter}{p.~9}
  \row{現象の経過を表す量}{時間}{time}{$t$}{s}{second}{p.~9}
  \row{単位時間当たりに移動する距離}{速度}{velocity}{$v$}{m/s}{meter per second}{p.~9}
  \row{単位時間当たりに変化する速さ}{加速度}{acceleration}{$a$}{m/s$^2$}{meter per second per second}{p.~13}
  \row{物体が落下するときの加速度(9.8m/s$^2$)}{重力加速度}{acceleration of gravity}{$g$}{m/s$^2$}{meter per second per second}{p.~25}
\Hline
\end{tabular}
\vfil

\textbf{第2章 力と運動}\\
\begin{tabular}{\ppppppp}\Hline
  \hfil 意味	&\hfil 科学量(日本語)	&\hfil 科学量(英語)&\hfil 量記号	&\hfil 単位記号	&\hfil 単位(英語)	&\hfil 教科書 \\\hline
  \row{物体の運動を変化させる作用}{力}{force}{$F, f$}{N}{newton}{p.~34}
  \row{地球が物体を引く力}{重力}{gravity, weight}{$W, w$}{N}{newton}{p.~44}
  \row{物体の移動のしにくさ}{質量}{mass	}{$M, m$}{kg}{kilogram}{p.~44, 59}
  \row{ばねを単位長さだけ伸縮させる力}{ばね定数}{spring constant}{$k$}{N/m}{newton per meter}{p.~47}
  \row{接触面から垂直に働く力}{垂直抗力}{normal reaction}{$N$}{N}{newton}{p.~49}
  \row{接触面で水平に働く抵抗力}{摩擦力}{friction}{$f$}{N}{newton}{p.~49}
  \row{摩擦力と垂直抗力との比}{摩擦係数}{coefficient of friction}{$\mu$}{---}{---}{p.~51}
  \row{接続点から糸の向きに働く力}{張力}{tension}{$T$}{N}{newton}{p.~65}
\Hline
\end{tabular}
\vfil

\textbf{第3章 運動量保存則}\\
\begin{tabular}{\ppppppp}\Hline
  \hfil 意味	&\hfil 科学量(日本語)	&\hfil 科学量(英語)&\hfil 量記号	&\hfil 単位記号	&\hfil 単位(英語)	&\hfil 教科書 \\\hline
  \row{力と時間との積}{力積}{impulse}{$I$}{N$\cdot$s}{newton second}{p.~77}
  \row{運動の勢い(速さと質量の積)}{運動量}{momentum}{$p$}{kg$\cdot$m/s}{kilogram meter per second}{p.~78}
  \row{跳ね返りやすさ}{反発係数}{coefficent of restitution}{$e$}{---}{---}{p.~87, 91}
\Hline
\end{tabular}


\newpage
\textbf{第4章 力学的エネルギー保存則}\\
\begin{tabular}{\ppppppp}\Hline
  \hfil 意味	&\hfil 科学量(日本語)	&\hfil 科学量(英語)&\hfil 量記号	&\hfil 単位記号	&\hfil 単位(英語)	&\hfil 教科書 \\\hline
  \row{力とその力の向きに動いた距離との積}{仕事}{work}{$W$}{J}{joule}{p.~94}
  \row{単位時間あたりにする仕事}{仕事率}{power}{$P$}{W}{watt}{p.~98}
  \row{仕事をすることができる能力}{エネルギー}{energy}{$E, U, K$}{J}{joule}{p.~100}
  \row{物体の運動に伴うエネルギー}{運動エネルギー}{kinetic energy}{$K$}{J}{joule}{p.~100}
  \row{位置の変化に伴うエネルギー}{位置エネルギー}{potential energy}{$U$}{J}{joule}{p.~104}
\Hline
\end{tabular}
\vfil

\textbf{第5章 等速円運動と単振動}\\
\begin{tabular}{\ppppppp}\Hline
  \hfil 意味	&\hfil 科学量(日本語)	&\hfil 科学量(英語)&\hfil 量記号	&\hfil 単位記号	&\hfil 単位(英語)	&\hfil 教科書 \\\hline
  \row{中心からの距離}{半径}{radius}{$r$}{m}{meter}{p.~118}
  \row{単位時間当たりに回転する角度}{角速度}{angular velocity}{$\omega$}{rad/s}{radian per seond}{p.~119}
  \row{一回転にかかる時間}{周期}{period}{$T$}{s}{second}{p.~119}
  \row{単位時間当たりに回転する回数}{回転数}{number of revolutions}{$n$}{Hz}{hertz}{p.~119}
  \row{単位時間当たりに変化する位相}{角振動数}{circular frequency}{$\omega$}{rad/s}{radian per second}{p.~128}
\Hline
\end{tabular}
\vfil

\textbf{第6章 万有引力の法則}\\
\begin{tabular}{\ppppppp}\Hline
  \hfil 意味	&\hfil 科学量(日本語)	&\hfil 科学量(英語)&\hfil 量記号	&\hfil 単位記号	&\hfil 単位(英語)	&\hfil 教科書 \\\hline
  \row{重力相互作用に関する定数}{万有引力定数}{constant of gravitation}{$G$}{N$\cdot$kg$^2$/m$^2$}{略}{p.~144}
\Hline
\end{tabular}
\vskip2zh


*英語の綴りは、電子辞書で発音も確認してください.

*量記号は斜体(イタリック体)で書きます.

*単位記号は立体(ローマン体)で書きます.

*人命由来の単位記号は大文字で書き始めます.しかし、綴りの時には小文字で書きます.


\end{document}
