\documentclass[b5j, landscape]{jarticle}

\special{dviout -y=B5L}

\pagestyle{empty}
\topmargin=-20mm
%\oddsidemargin=-5mm
%\textwidth=200mm
\textheight=150mm

\begin{document}

\centerline{\LARGE 量記号と単位(2011年度 第2学年 後期中間試験用)}

\vskip1zh

\begin{tabular}{|l|c|l|c|c|l|}\hline
\hfil 意味	&科学量(日本語)				&\hfil 科学量(英語)				&量記号				&単位記号		&\hfil 単位(英語)					\\ \hline
中心から円周上の点までの距離			&半径			&radius						&$R, r$		&m		&meter							\\ \hline
ものの移動のしにくさ	&質量	&mass	&$M, m$	&kg	&kilogram	\\ \hline
現象の経過を表す量	&時間	&time	&$t$	&s	&second	\\ \hline
単位時間当たりに移動する距離	&速度	&velocity	&$v$	&m/s	&meter per second	\\ \hline
単位時間当たりに変化する速さ			&加速度			&acceleration				&$a$		&m/s$^2$	&meter per second per second	\\ \hline
ものを移動させる作用					&力				&force						&$F, f$		&N		&newton							\\ \hline
移動する運動の程度(速さと質量の積)	&運動量			&momentum					&$p$		&kg$\cdot$m/s	&kilogram meter per second		\\ \hline
力と時間との積							&力積			&impulse					&$I$		&N$\cdot$s		&newton second					\\ \hline
力とその力の向きに動いた距離との積		&仕事			&work						&$W$		&J		&joule							\\ \hline
仕事をすることができる能力				&エネルギー		&energy						&$E, U, K$	&J		&joule							\\ \hline
\end{tabular}
\vskip10mm

*英語の綴りは、電子辞書で発音も確認してください。

*量記号は斜体(イタリック体)で書きます。

*単位記号は立体(ローマン体)で書きます。

*人命由来の単位記号は大文字で書き始めます。しかし、綴りの時には小文字で書きます。

*十の内、四つ程度出題します。



\end{document}
