\documentclass[b5j, landscape]{jarticle}

\special{dviout -y=B5L}

\pagestyle{empty}
\topmargin=-20mm
\oddsidemargin=-10mm
\textwidth=200mm
\textheight=190mm

\begin{document}

\centerline{\LARGE 量記号と単位(3学年用)}

\vskip1zh

\begin{tabular}{|l|c|l|c|c|l|}\hline
     意\hfill 味     			&科学量				&科学量英名				&量記号		&単位		&単位英名			\\ \hline
位置の変化を表す量						&変位				&displacement			&$x, y$		&(m)		&meter				\\ \hline
慣性の大きさを表す量					&質量				&mass					&$M, m$		&(kg)		&kilogram			\\ \hline
自然現象の経過を記述するための変数		&時間				&time					&$t$		&(s)		&second				\\ \hline
交差する二直線の広がり具合				&角度				&angle					&$\theta$	&(rad)		&radian				\\ \hline
物体を変形・運動を変化させる作用		&力					&force					&$F, f$		&(N)		&newton				\\ \hline
力とその力の向きに動いた距離との積		&仕事				&work					&$W$		&(J)		&joule				\\ \hline
仕事をすることができる能力				&エネルギー			&energy					&$E$		&(J)		&joule				\\ \hline
単位時間当たりに繰り返す回数			&周波数(振動数)	&frequency				&$f$		&(Hz)		&Hertz				\\ \hline
帯びている電気の量						&電気量(電荷)		&electric charge		&$q, Q$		&(C)		&coulomb			\\ \hline
電荷に力が生じさせる空間				&電場				&electric field			&$E$		&(N/C)		&newton per coulomb	\\ \hline
電場の中で電荷が持つ位置エネルギー		&電位				&electric potential		&$V$		&(V)		&volt				\\ \hline
電荷を蓄える能力を表す量				&電気容量			&capacitance			&$C$		&(F)		&farad				\\ \hline
電荷の流れ								&電流				&current				&$I, i$		&(A=C/s)	&ampere				\\ \hline
電流の流れにくさ						&電気抵抗			&electric resistance	&$r$		&($\Omega$)	&ohm				\\ \hline
物質によって定まる抵抗の係数			&抵抗率				&resistivity			&$\rho$		&($\Omega$m)	&ohm meter		\\ \hline
単位時間当たりに消費される電気エネルギー	&電力			&electric power			&$P$		&(W=J/s)&watt					\\ \hline
N極からS極へつながる仮想的な紐			&磁束				&magnetic flux			&$\mathit{\Phi}$		&(Wb)	&weber		\\ \hline
磁束の密度								&磁束密度			&magnetic flux density	&$B$		&(T)		&tesla				\\ \hline
コイルに交流を流した時の流れ難さ		&自己インダクタンス	&self-inductance		&$L$		&(H)		&henry				\\ \hline
二つの回路の電磁気的なつながりを示す量	&相互インダクタンス	&mutual inductance		&$M$		&(H)		&henry				\\ \hline
交流回路における抵抗(複素数の抵抗)		&インピーダンス		&inpedance				&$Z$		&($\Omega$)	&ohm				\\ \hline
\end{tabular}
\end{document}




