\documentclass[b5j, landscape]{jarticle}

\special{dviout -y=B5L}

\pagestyle{empty}
\topmargin=-20mm
\oddsidemargin=-5mm
\textwidth=200mm
\textheight=190mm

\begin{document}

\centerline{\LARGE 量記号と単位(2学年用)}

\vskip1zh

\begin{tabular}{|l|c|l|c|c|l|}\hline
     意\hfill 味     			&科学量			&科学量英名					&量記号		&単位		&単位英名						\\ \hline
空間的な隔たり							&距離			&space						&$s$		&(m)		&meter							\\ \hline
物体の2点間の距離						&長さ			&length						&$\ell$		&(m)		&meter							\\ \hline
中心から円周上の点までの距離			&半径			&radius						&$R, r$		&(m)		&meter							\\ \hline
ものの移動しにくさ						&質量			&mass						&$M, m$		&(kg)		&kilogram						\\ \hline
現象の経過を表す変数					&時間			&time						&$t$		&(s)		&second							\\ \hline
交差する2直線の広がり具合				&角度			&angle						&$\theta$	&(rad)		&radian							\\ \hline
ものの大きさ							&体積			&volume						&$V$		&(m$^3$)	&cubic meter					\\ \hline
単位時間当たりに移動する距離			&速度(速さ)	&velocity					&$v$		&(m/s)		&meter per second				\\ \hline
単位時間当たりに変化する速さ			&加速度			&acceleration				&$a$		&(m/s$^2$)	&meter per second per second	\\ \hline
ものが落下するときの加速度(9.8m/s$^2$)	&重力加速度		&acceleration of gravity	&$g$		&(m/s$^2$)	&meter per second per second	\\ \hline
ものを移動させる作用					&力				&force						&$F, f$		&(N)		&newton							\\ \hline
地球の引力								&重力			&gravity					&$Mg, mg$	&(N)		&newton							\\ \hline
接触面から垂直に働く力					&垂直抗力		&normal reaction			&$N$		&(N)		&newton							\\ \hline
接触面で水平に働く抵抗力				&摩擦力			&friction					&$f$		&(N)		&newton							\\ \hline
接続点から糸の向きに働く力				&張力			&tension					&$T$		&(N)		&newton							\\ \hline
バネを単位長さだけ伸縮させる力			&バネ定数		&spring constant			&$k$		&(N/m)		&newton per meter				\\ \hline
摩擦力と垂直抗力との比					&摩擦係数		&coefficient of friction	&$\mu$		&---		&---							\\ \hline
ものを回転させる作用(別名トルク)		&力のモーメント	&moment of force			&$M$		&(Nm)		&newton meter					\\ \hline
移動する運動の程度(速さと質量の積)	&運動量			&momentum					&$p$		&(kg$\cdot$m/s)	&kilogram meter per second		\\ \hline
力と時間との積							&力積			&impulse					&$I$		&(N$\cdot$s)		&newton second					\\ \hline
力とその力の向きに動いた距離との積		&仕事			&work						&$W$		&(J)		&joule							\\ \hline
仕事をすることができる能力				&エネルギー		&energy						&$E, U, K$	&(J)		&joule							\\ \hline
温度変化で物体から出入りするもの				&熱		&heat						&$Q, q$	&(J)		&joule							\\ \hline
冷熱の指標				&温度		&temperature						&$T$	&(K)		&kelvin							\\ \hline
\end{tabular}
\end{document}


