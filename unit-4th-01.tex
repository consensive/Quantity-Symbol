\documentclass[a4j, landscape]{jarticle}

\special{dviout -y=A4L}

\pagestyle{empty}
\topmargin=-20mm
\oddsidemargin=-5mm
\textwidth=200mm
\textheight=190mm

\begin{document}

\centerline{\LARGE 量記号と単位(4学年用)}

\vskip1zh

\begin{tabular}{|l|c|l|c|c|l|}\hline
     意\hfill 味     
	& 科 学 量 
	& 科 学 量 英 名
	& 量記号 
	&単位記号
	&単位英語名						\\ \hline
空間的な隔たり							&距離			&space						&$s$		&m		&meter							\\ \hline
物体の2点間の距離						&長さ			&length						&$\ell$		&m		&meter							\\ \hline
位置のズレ							&変位				&displacement		&$x, y, u$		&m		&meter							\\ \hline
中心から円周上の点までの距離			&半径			&radius						&$R, r$		&m		&meter							\\ \hline
現象の経過を表す変数				&時間				&time				&$t$			&s		&second							\\ \hline
交差する2直線の広がり具合				&角度			&angle						&$\theta$	&rad		&radian							\\ \hline
図形の大きさ・広さ		&面積			&area						&$A, a, S$		&m$^2$	&square meter					\\ \hline
物体が空間を占める度合い							&体積			&volume						&$V$		&m$^3$	&cubic meter					\\ \hline
物体の移動のしにくさ					&質量				&mass				&$M, m$			&kg		&kilogram						\\ \hline

単位長さ当たりの質量				&線密度				&linear density			&$\lambda$			&kg/m	&kilogram per meter		\\ \hline
単位面積当たりの質量				&面密度				&surface density			&$\sigma$			&kg/m$^2$	&kilogram per square meter		\\ \hline
単位体積当たりの質量				&密度				&density			&$\rho$			&kg/m$^3$	&kilogram per cubic meter		\\ \hline


単位時間当たりに移動する距離			&速度(速さ)	&velocity					&$v$		&m/s		&meter per second				\\ \hline
単位時間当たりに変化する速さ			&加速度			&acceleration				&$a$		&m/s$^2$	&meter per second per second	\\ \hline
地上で物体が自由落下するときの加速度	&重力加速度		&acceleration of gravity	&$g$		&m/s$^2$	&meter per second per second	\\ \hline
\end{tabular}



\begin{tabular}{|l|c|l|c|c|l|}\hline
     意\hfill 味     		&科学量				&科学量英名			&量記号			&単位記号	&単位英語名						\\ \hline
ものを移動させる作用					&力				&force						&$F, f$		&N		&newton							\\ \hline
地球の引力								&重力			&gravity					&$Mg, mg$	&N		&newton							\\ \hline
接触面から垂直に働く力					&垂直抗力		&normal reaction			&$N$		&N		&newton							\\ \hline
接触面で水平に働く抵抗力				&摩擦力			&friction					&$f$		&N		&newton							\\ \hline
接続点から糸の向きに働く力				&張力			&tension					&$T$		&N		&newton							\\ \hline
バネを単位長さだけ伸縮させる力			&バネ定数		&spring constant			&$k$		&N/m		&newton per meter				\\ \hline
摩擦力と垂直抗力との比					&摩擦係数		&coefficient of friction	&$\mu$		&---		&---							\\ \hline

ものを回転させる作用(別名トルク)		&力のモーメント	&moment of force			&$M$		&Nm		&newton meter					\\ \hline

移動する運動の程度(速さと質量の積)	&運動量			&momentum					&$p$		&kg$\cdot$m/s	&kilogram meter per second		\\ \hline
力と時間との積							&力積			&impulse					&$I$		&N$\cdot$s		&newton second					\\ \hline
力とその力の向きに動いた距離との積		&仕事			&work						&$W$		&J		&joule							\\ \hline
仕事をすることができる能力				&エネルギー		&energy						&$E, U, K$	&J		&joule							\\ \hline
温度変化で物体から出入りするもの				&熱		&heat						&$Q, q$	&J		&joule							\\ \hline
冷熱の指標				&温度		&temperature						&$T$	&K		&kelvin							\\ \hline
位置の変化を表す量						&変位				&displacement			&$x, y$		&m		&meter				\\ \hline
慣性の大きさを表す量					&質量				&mass					&$M, m$		&kg		&kilogram			\\ \hline
自然現象の経過を記述するための変数		&時間				&time					&$t$		&s		&second				\\ \hline
交差する二直線の広がり具合				&角度				&angle					&$\theta$	&rad		&radian				\\ \hline
物体を変形・運動を変化させる作用		&力					&force					&$F, f$		&N		&newton				\\ \hline
力とその力の向きに動いた距離との積		&仕事				&work					&$W$		&J		&joule				\\ \hline
仕事をすることができる能力				&エネルギー			&energy					&$E$		&J		&joule				\\ \hline
単位時間当たりに繰り返す回数			&周波数(振動数)	&frequency				&$f$		&Hz		&Hertz				\\ \hline
帯びている電気の量						&電気量(電荷)		&electric charge		&$q, Q$		&C		&coulomb			\\ \hline
電荷に力が生じさせる空間				&電場				&electric field			&$E$		&N/C		&newton per coulomb	\\ \hline
電場の中で電荷が持つ位置エネルギー		&電位				&electric potential		&$V$		&V		&volt				\\ \hline
電荷を蓄える能力を表す量				&電気容量			&capacitance			&$C$		&F		&farad				\\ \hline
電荷の流れ								&電流				&current				&$I, i$		&A=C/s	&ampere				\\ \hline
電流の流れにくさ						&電気抵抗			&electric resistance	&$r$		&$\Omega$	&ohm				\\ \hline
物質によって定まる抵抗の係数			&抵抗率				&resistivity			&$\rho$		&$\Omega$m	&ohm meter		\\ \hline
単位時間当たりに消費される電気エネルギー	&電力			&electric power			&$P$		&W=J/s&watt					\\ \hline
N極からS極へつながる仮想的な紐			&磁束				&magnetic flux			&$\mathit{\Phi}$		&Wb	&weber		\\ \hline
磁束の密度								&磁束密度			&magnetic flux density	&$B$		&T		&tesla				\\ \hline
コイルに交流を流した時の流れ難さ		&自己インダクタンス	&self-inductance		&$L$		&H		&henry				\\ \hline
二つの回路の電磁気的なつながりを示す量	&相互インダクタンス	&mutual inductance		&$M$		&H		&henry				\\ \hline
交流回路における抵抗(複素数の抵抗)		&インピーダンス		&inpedance				&$Z$		&$\Omega$	&ohm				\\ \hline
物を移動させる作用					&力					&force				&$F, f$			&N		&newton							\\ \hline
地球が中心に引く力($g$=9.8m/s$^2$)	&重力				&gravity			&$mg, Mg$		&N		&newton							\\ \hline
接続点から糸の向きに働く力			&張力				&tension			&$T$			&N		&newton							\\ \hline
接触面から垂直に働く力				&垂直抗力			&normal reaction	&$N$			&N		&newton							\\ \hline
接触面で水平に働く抵抗力			&摩擦力				&friction			&$f$			&N		&newton							\\ \hline
ものを回転させる作用(別名:トルク)&力のモーメント		&moment of force	&$M$			&N$\cdot$m		&newton meter					\\ \hline
ものの回転のしにくさ				&慣性モーメント		&moment of inertia	&$I$			&kg m$^2$	&kilogram square meter			\\ \hline
単位時間当たりに回転する角度		&角速度				&angular velocity	&$\omega$		&rad/s	&radian per second				\\ \hline
単位面積当たりに加わる力			&圧力				&pressure			&$p$			&Pa		&pascal							\\ \hline
単位時間当たりに繰り返す回数		&周波数(振動数)	&frequency			&$f, \nu$		&Hz		& hertz							\\ \hline
波の1周期分の長さ					&波長				&wave length		&$\lambda$		&m		&meter							\\ \hline
冷熱の指標							&温度				&temperature		&$T, t$			&K		&kelvin							\\ \hline
仕事をすることができる能力			&エネルギー			&energy				&$E$			&J		&joule							\\ \hline
均一さ,乱雑さの度合い				&エントロピー		&entropy			&$S$			&J/K		&joule per kelvin				\\ \hline
\end{tabular}
\end{document}








\newpage


\centerline{\LARGE 量記号と単位(4学年用 No. 2)}

\vskip1zh

\begin{tabular}{|c|c|c|c|c|l|}\hline



距離,長さ		&length							&$\ell, s, d$	&m			&meter			&空間的な隔たり \\ \hline
高さ			&height							&$h$			&m			&meter			&垂直方向の距離 \\ \hline
焦点距離		&focal length					&$f$			&m			&meter			&\\ \hline
振幅			&amplitude						&$A$			&m			&meter			&振動の大きさ \\ \hline
仕事			&work							&$W$			&J			&joule			&力とその力の向きに動いた距離との積 \\ \hline
位置エネルギー	&gravitational potential energy	&$U_\mathrm{G}$	&J			&joule			&高さと重力で決まるエネルギー \\ \hline
弾性エネルギー	&elastic potential energy		&$U_\mathrm{E}$	&J			&joule			&バネの弾力によって得られるエネルギー \\ \hline
運動エネルギー	&kinetic energy					&$K$			&J			&joule			&速さと質量で決まるエネルギー \\ \hline
回転エネルギー	&rotational kinetic energy		&$K_\mathrm{R}$	&J			&joule			&角速度と慣性モーメントで決まるエネルギー \\ \hline
内部エネルギー	&internal energy				&$U$			&J			&joule			&物質内部のミクロのエネルギー \\ \hline
熱量			&quantity of heat				&$Q$			&J			&joule			&温度を上昇させるもとになるエネルギー \\ \hline
科学量				&科学量英名					&量記号				&単位記号	&単位英語名						&補足\\ \hline\hline
摩擦係数			&coefficient of friction	&$\mu$				& ---		& ---							& 摩擦力と垂直抗力の比\\ \hline
屈折率				&index of refraction		&$n$				& ---		& ---							& 二つの媒質中の波の速度の比\\ \hline
歪み				&strain						&$\varepsilon$		& ---		& ---							& 単位長さ当たりの伸縮率 \\ \hline
角度				&angle						&$\theta, \alpha$	&rad		& radian						& 交差する2直線の広がり具合 \\ \hline
面積				&area						&$S$				&m$^2$		& square meter					& ものの広さ \\ \hline
線密度				&linear density				&$\sigma$			&kg/m		&kilogram per meter				& 単位長さ当たりの質量 \\ \hline
重力加速度			&acceleration of gravity	&$g$				&m/s$^2$	&meter per second per second	& ものが落下する時の加速度 \\ \hline
バネ定数			&spring constant			&$k$				&N/m		&newton per meter				& バネを単位長さ伸縮させる力 \\ \hline
ヤング率			&Young's modulus			&$E$				&N/m$^2$	&newton per square meter		& 応力と歪の比 \\ \hline
物質量				&amount of substance		&$n$				&mol		&mole							& 物質の構成要素の粒子の量 \\ \hline
気体定数			&gas constant				&$R$				&J/mol K	&joule per mol per kelvin		& 気体状態方程式の比例係数 \\ \hline
モル比熱			&molar heat					&$c$				&J/mol K	&joule per mol per kelvin		& 単位物質量当たりの温まりにくさ \\ \hline
\end{tabular}
