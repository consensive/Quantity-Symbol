\documentclass[b5j, landscape]{jarticle}

\special{dviout -y=B5L}

\pagestyle{empty}
\topmargin=-20mm
%\oddsidemargin=-5mm
%\textwidth=200mm
\textheight=150mm

\begin{document}

\centerline{\LARGE 量記号と単位(2011年度 第2学年 前期中間試験用)}

\vskip1zh

\begin{tabular}{|l|c|l|c|c|l|}\hline
\hfil 意味	&科学量(日本語)				&\hfil 科学量(英語)				&量記号				&単位記号		&\hfil 単位(英語)					\\ \hline
二点間の隔たり	&長さ	&length	&$\ell$	&m	&meter	\\ \hline
立体が占める空間の大きさ	&体積	&volume	&$V$	&m$^3$	&cubic meter	\\ \hline
物体の移動のしにくさ	&質量	&mass	&$M, m$	&kg	&kilogram	\\ \hline
現象の経過を表す量	&時間	&time	&$t$	&s	&second	\\ \hline
単位時間当たりに移動する距離	&速度(速さ)	&velocity	&$v$	&m/s	&meter per second	\\ \hline
接続点から糸の向きに働く力	&張力	&tension	&$T$	&N	&newton	\\ \hline
バネを単位長さだけ伸縮させる力	&バネ定数	&spring constant	&$k$	&N/m	&newton per meter	\\ \hline
ものを回転させる作用(別名トルク)	&力のモーメント	&moment of force	&$M$	&N$\cdot$m	&newton meter	\\ \hline
\end{tabular}

*英語の綴りは、電子辞書で発音も確認してください。

*量記号は斜体(イタリック体)で書きます。

*単位記号は立体(ローマン体)で書きます。

*人命由来の単位記号は大文字で書き始めます。しかし、綴りの時には小文字で書きます。

*八つの内、四つ程度出題します。


\vfill
\begin{tabular}{|l|c|l|c|c|l|}\hline
\hfil 意味	&科学量(日本語)				&\hfil 科学量(英語)				&量記号				&単位記号		&\hfil 単位(英語)					\\ \hline
二点間の隔たり	&長さ	&length	&$\ell$	&m	&meter	\\ \hline
立体が占める空間の大きさ	&体積	&volume	&$V$	&m$^3$	&cubic meter	\\ \hline
物体の移動のしにくさ	&質量	&mass	&$M, m$	&kg	&kilogram	\\ \hline
現象の経過を表す量	&時間	&time	&$t$	&s	&second	\\ \hline
単位時間当たりに移動する距離	&速度(速さ)	&velocity	&$v$	&m/s	&meter per second	\\ \hline
接続点から糸の向きに働く力	&張力	&tension	&$T$	&N	&newton	\\ \hline
バネを単位長さだけ伸縮させる力	&バネ定数	&spring constant	&$k$	&N/m	&newton per meter	\\ \hline
ものを回転させる作用(別名トルク)	&力のモーメント	&moment of force	&$M$	&N$\cdot$m	&newton meter	\\ \hline
\end{tabular}


\end{document}
