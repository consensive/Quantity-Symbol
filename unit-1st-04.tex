\documentclass[b5j, landscape]{jarticle}

\special{dviout -y=B5L}

\pagestyle{empty}
\topmargin=-20mm
\oddsidemargin=-10mm
%\oddsidemargin=-5mm
%\textwidth=200mm
\textheight=150mm

\begin{document}

\centerline{\LARGE 量記号と単位(2011年度 第1学年 学年末試験用)}

\vskip1zh

\begin{tabular}{|l|c|l|c|c|l|l|}\hline
\hfil 意味	&科学量(日本語)				&\hfil 科学量(英語)				&量記号				&単位記号		&\hfil 単位(英語)					&\hfil 単位(日本語)\\ \hline
円の中心から周までの距離	&半径				&radius					&$r$				&m		&meter						&メートル \\ \hline
立体が占める空間の大きさ	&体積				&volume					&$V, v$				&m$^3$	&cubic meter				&立方メートル \\ \hline
物体の移動のしにくさ	&質量				&mass					&$M, m$				&kg		&kilogram					&キログラム \\ \hline
現象の経過を表す量	&時間				&time					&$T, t$				&s		&second						&秒 \\ \hline
接触面から垂直に働く力					&垂直抗力		&normal reaction			&$N$		&N		&newton	&ニュートン \\ \hline
接触点から糸の向きに働く力	&張力				&tension 		&$T$			&N		&newton						&ニュートン \\ \hline
ばねを単位長さだけ伸縮させる力	&ばね定数				&spring constant		&$k$			&N/m		&newton per meter						&ニュートン毎メートル \\ \hline
摩擦力と垂直抗力との比					&摩擦係数		&coefficient of friction	&$\mu$		&---		&---	&---	\\ \hline
\end{tabular}

*英語の綴りは、電子辞書で発音も確認してください。

*量記号は斜体(イタリック体)で書きます。

*単位記号は立体(ローマン体)で書きます。

*人命由来の単位記号は大文字で書き始めます。しかし、綴りの時には小文字で書きます。

*八つの内、四つ程度出題します。


\vfill
\begin{tabular}{|l|c|l|c|c|l|l|}\hline
\hfil 意味	&科学量(日本語)				&\hfil 科学量(英語)				&量記号				&単位記号		&\hfil 単位(英語)					&\hfil 単位(日本語)\\ \hline
円の中心から周までの距離	&半径				&radius					&$r$				&m		&meter						&メートル \\ \hline
立体が占める空間の大きさ	&体積				&volume					&$V, v$				&m$^3$	&cubic meter				&立方メートル \\ \hline
物体の移動のしにくさ	&質量				&mass					&$M, m$				&kg		&kilogram					&キログラム \\ \hline
現象の経過を表す量	&時間				&time					&$T, t$				&s		&second						&秒 \\ \hline
接触面から垂直に働く力					&垂直抗力		&normal reaction			&$N$		&N		&newton	&ニュートン \\ \hline
接触点から糸の向きに働く力	&張力				&tension 		&$T$			&N		&newton						&ニュートン \\ \hline
ばねを単位長さだけ伸縮させる力	&ばね定数				&spring constant		&$k$			&N/m		&newton per meter						&ニュートン毎メートル \\ \hline
摩擦力と垂直抗力との比					&摩擦係数		&coefficient of friction	&$\mu$		&---		&---	&---	\\ \hline
\end{tabular}

\end{document}

