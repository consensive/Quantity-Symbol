\documentclass[b5j, landscape]{jarticle}

\special{dviout -y=B5L}

\pagestyle{empty}
\topmargin=-20mm
%\oddsidemargin=-5mm
%\textwidth=200mm
\textheight=150mm

\begin{document}

\centerline{\LARGE 量記号と単位(2011年度 第1学年 後期中間試験用)}

\vskip1zh

\begin{tabular}{|l|c|l|c|c|l|l|}\hline
\hfil 意味	&科学量(日本語)				&\hfil 科学量(英語)				&量記号				&単位記号		&\hfil 単位(英語)					&\hfil 単位(日本語)\\ \hline
円の中心から周までの距離	&半径				&radius					&$r$				&m		&meter						&メートル \\ \hline
垂直方向の長さ	&高さ				&height					&$h$				&m		&meter						&メートル \\ \hline
立体が占める空間の大きさ	&体積				&volume					&$V, v$				&m$^3$	&cubic meter				&立方メートル \\ \hline
物体の移動のしにくさ	&質量				&mass					&$M, m$				&kg		&kilogram					&キログラム \\ \hline
現象の経過を表す量	&時間				&time					&$T, t$				&s		&second						&秒 \\ \hline
1回の振動に要する時間	&周期				&period			&$T$				&s		&second						&秒 \\ \hline
単位時間辺りに繰り返す回数	&振動数(周波数)	&frequency			&$f$				&Hz		&hertz						&ヘルツ \\ \hline
変位の最大値	&振幅				&amplitude		&$A$				&m		&meter						&メートル \\ \hline
波の1周期分の長さ	&波長				&wave length		&$\lambda$			&m		&meter						&メートル \\ \hline
\end{tabular}

*英語の綴りは、電子辞書で発音も確認してください。

*量記号は斜体(イタリック体)で書きます。

*単位記号は立体(ローマン体)で書きます。

*人命由来の単位記号は大文字で書き始めます。しかし、綴りの時には小文字で書きます。

*九つの内、四つ程度出題します。


\vfill
\begin{tabular}{|l|c|l|c|c|l|l|}\hline
\hfil 意味	&科学量(日本語)				&\hfil 科学量(英語)				&量記号				&単位記号		&\hfil 単位(英語)					&\hfil 単位(日本語)\\ \hline
円の中心から周までの距離	&半径				&radius					&$r$				&m		&meter						&メートル \\ \hline
垂直方向の長さ	&高さ				&height					&$h$				&m		&meter						&メートル \\ \hline
立体が占める空間の大きさ	&体積				&volume					&$V, v$				&m$^3$	&cubic meter				&立方メートル \\ \hline
物体の移動のしにくさ	&質量				&mass					&$M, m$				&kg		&kilogram					&キログラム \\ \hline
現象の経過を表す量	&時間				&time					&$T, t$				&s		&second						&秒 \\ \hline
1回の振動に要する時間	&周期				&period			&$T$				&s		&second						&秒 \\ \hline
単位時間辺りに繰り返す回数	&振動数(周波数)	&frequency			&$f$				&Hz		&hertz						&ヘルツ \\ \hline
変位の最大値	&振幅				&amplitude		&$A$				&m		&meter						&メートル \\ \hline
波の1周期分の長さ	&波長				&wave length		&$\lambda$			&m		&meter						&メートル \\ \hline
\end{tabular}


\end{document}
