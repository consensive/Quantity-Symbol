\documentclass[a4paper,papersize,10pt,landscape]{jsarticle}



\def\row#1#2#3#4#5#6#7{#1 & #2 & #3 &\hfil #4 &\hfil #5 & #6 & #7 \\ }
\def\ppppppp{p{70mm}p{30mm}p{40mm}p{14mm}p{18mm}p{50mm}p{15mm}}
\makeatletter
\def\Hline{%
\noalign{\ifnum0=`}\fi\hrule \@height 2pt \futurelet
\reserved@a\@xhline}
\makeatother


\pagestyle{empty}
\topmargin=-10mm
\oddsidemargin=-10mm
\textwidth=270mm
\textheight=170mm

\begin{document}

\centerline{\LARGE 量記号と単位(力学I)}

\vskip1zh

\textbf{第1章 物体の運動}\\
\begin{tabular}{\ppppppp}\Hline
  \hfil 意味	&\hfil 科学量(日本語)	&\hfil 科学量(英語)&\hfil 量記号	&\hfil 単位記号	&\hfil 単位(英語)	&\hfil 教科書 \\\hline
  \row{物体の2点間の距離}{距離}{distance}{$x$}{m}{meter}{p.~9}
  \row{現象の経過を表す量}{時間}{time}{$t$}{s}{second}{p.~9}
  \row{単位時間当たりに移動する距離}{速度}{velocity}{$v$}{m/s}{meter per second}{p.~9}
  \row{単位時間当たりに変化する速さ}{加速度}{acceleration}{$a$}{m/s$^2$}{meter per second per second}{p.~13}
  \row{物体が落下するときの加速度(9.8m/s$^2$)}{重力加速度}{acceleration of gravity}{$g$}{m/s$^2$}{meter per second per second}{p.~25}
\Hline
\end{tabular}
\vskip1zh
\vfil

\textbf{第2章 力と運動}\\
\begin{tabular}{\ppppppp}\Hline
  \hfil 意味	&\hfil 科学量(日本語)	&\hfil 科学量(英語)&\hfil 量記号	&\hfil 単位記号	&\hfil 単位(英語)	&\hfil 教科書 \\\hline
  \row{物体の運動を変化させる作用}{力}{force}{$F, f$}{N}{newton}{p.~34}
  \row{地球が物体を引く力}{重力}{gravity, weight}{$W, w$}{N}{newton}{p.~44}
  \row{物体の移動のしにくさ}{質量}{mass	}{$M, m$}{kg}{kilogram}{p.~44, 59}
  \row{ばねを単位長さだけ伸縮させる力}{ばね定数}{spring constant}{$k$}{N/m}{newton per meter}{p.~47}
  \row{接触面から垂直に働く力}{垂直抗力}{normal reaction}{$N$}{N}{newton}{p.~49}
  \row{接触面で水平に働く抵抗力}{摩擦力}{friction}{$f$}{N}{newton}{p.~49}
  \row{摩擦力と垂直抗力との比}{摩擦係数}{coefficient of friction}{$\mu$}{---}{---}{p.~51}
  \row{接続点から糸の向きに働く力}{張力}{tension}{$T$}{N}{newton}{p.~65}
\Hline
\end{tabular}
\vskip1zh
\vfil

\textbf{第3章 運動量保存則}\\
\begin{tabular}{\ppppppp}\Hline
  \hfil 意味	&\hfil 科学量(日本語)	&\hfil 科学量(英語)&\hfil 量記号	&\hfil 単位記号	&\hfil 単位(英語)	&\hfil 教科書 \\\hline
  \row{力と時間との積}{力積}{impulse}{$I$}{N$\cdot$s}{newton second}{p.~77}
  \row{移動する運動の勢い(速さと質量の積)}{運動量}{momentum}{$p$}{kg$\cdot$m/s}{kilogram meter per second}{p.~78}
  \row{跳ね返りやすさ}{反発係数}{coefficent of restitution}{$e$}{---}{---}{p.~87, 91}
\Hline
\end{tabular}
\vskip1zh
\vfil

\textbf{第4章 力学的エネルギー保存則}\\
\begin{tabular}{\ppppppp}\Hline
  \hfil 意味	&\hfil 科学量(日本語)	&\hfil 科学量(英語)&\hfil 量記号	&\hfil 単位記号	&\hfil 単位(英語)	&\hfil 教科書 \\\hline
  \row{力とその力の向きに動いた距離との積}{仕事}{work}{$W$}{J}{joule}{p.~94}
  \row{単位時間あたりにする仕事}{仕事率}{power}{$P$}{W}{watt}{p.~98}
  \row{仕事をすることができる能力}{エネルギー}{energy}{$E, U, K$}{J}{joule}{p.~100}
  \row{物体の運動に伴うエネルギー}{運動エネルギー}{kinetic energy}{$K$}{J}{joule}{p.~100}
  \row{位置の変化に伴うエネルギー}{位置エネルギー}{potential energy}{$U$}{J}{joule}{p.~104}
\Hline
\end{tabular}
\vskip1zh
\vfil

\textbf{第5章 等速円運動と単振動}\\
\begin{tabular}{\ppppppp}\Hline
  \hfil 意味	&\hfil 科学量(日本語)	&\hfil 科学量(英語)&\hfil 量記号	&\hfil 単位記号	&\hfil 単位(英語)	&\hfil 教科書 \\\hline
  \row{力とその力の向きに動いた距離との積}{仕事}{work}{$W$}{J}{joule}{p.~94}
\Hline
\end{tabular}
\vskip1zh
\vfil

\textbf{第6章 万有引力の法則}\\
\begin{tabular}{\ppppppp}\Hline
  \hfil 意味	&\hfil 科学量(日本語)	&\hfil 科学量(英語)&\hfil 量記号	&\hfil 単位記号	&\hfil 単位(英語)	&\hfil 教科書 \\\hline
  \row{力とその力の向きに動いた距離との積}{仕事}{work}{$W$}{J}{joule}{p.~94}
\Hline
\end{tabular}
\vskip1zh
\vfil


*英語の綴りは、電子辞書で発音も確認してください.
*量記号は斜体(イタリック体)で書きます.

*単位記号は立体(ローマン体)で書きます.

*人命由来の単位記号は大文字で書き始めます.しかし、綴りの時には小文字で書きます.

\hfil
\begin{tabular}{|p{70mm}|p{35mm}|p{40mm}|p{20mm}|p{20mm}|p{50mm}|p{15mm}|}\hline
\hfil 意味	&科学量(日本語)	&\hfil 科学量(英語)	&量記号	&単位記号	&\hfil 単位(英語)\\ \hline
物体の2点間の距離	&長さ	&length	&$\ell$	&m	&meter \\ \hline
立体が占める空間の大きさ	&体積	&volume	&$V, v$	&m$^3$	&cubic meter \\ \hline
交差する2直線の広がり具合	&角度	&angle	&$\theta$	&rad	&radian \\ \hline
ものの大きさ	&体積	&volume	&$V$	&m$^3$	&cubic meter \\ \hline
\end{tabular}



\end{document}
