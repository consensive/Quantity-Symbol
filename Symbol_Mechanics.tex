\documentclass[a4paper, 10pt,landscape]{jsarticle}

\special{dviout -y=A4L}
\special{pdf: pagesize width 297truemm height 210truemm}


\pagestyle{empty}
%\topmargin=-20mm
%\oddsidemargin=-5mm
%\textwidth=200mm
%\textheight=150mm

\begin{document}

\centerline{\LARGE 量記号と単位(力学I)}

\vskip1zh

\hfil
\begin{tabular}{|l|c|l|c|c|l|l|}\hline
\hfil 意味	&科学量(日本語)	&\hfil 科学量(英語)	&量記号	&単位記号	&\hfil 単位(英語)\\ \hline
物体の2点間の距離	&長さ	&length	&$\ell$	&m	&meter \\ \hline
立体が占める空間の大きさ	&体積	&volume	&$V, v$	&m$^3$	&cubic meter \\ \hline
交差する2直線の広がり具合	&角度	&angle	&$\theta$	&rad	&radian \\ \hline
ものの大きさ	&体積	&volume	&$V$	&m$^3$	&cubic meter \\ \hline
\end{tabular}



\hfil
\begin{tabular}{|l|c|l|c|c|l|l|l|}\hline
  \hfil 意味	&科学量(日本語)	&\hfil 科学量(英語)	&量記号	&単位記号	&\hfil 単位(英語)	&教科書\\ \hline

  物体の2点間の距離	&距離	&distance	&$x$	&m	&meter	&p.~9 \\ \hline
  現象の経過を表す量	&時間	&time	&$t$	&s	&second	& p.~9 \\ \hline
  単位時間当たりに移動する距離	&速度	&velocity	&$v$	&m/s	&meter per second	& p.~9 \\ \hline
  単位時間当たりに変化する速さ	&加速度	&acceleration	&$a$	&m/s$^2$	&meter per second per second	& p.~13 \\ \hline
  物体が落下するときの加速度(9.8m/s$^2$)	&重力加速度	&acceleration of gravity	&$g$	&m/s$^2$	&meter per second per second	& p.~25 \\ \hline
  物体の運動を変化させる作用	&力	&force	&$F, f$	&N	&newton	& p.~34 \\ \hline
  地球が物体を引く力	&重力	&gravity, weight	&$W, w$	&N	&newton	& p.~44 \\ \hline
  物体の移動のしにくさ	&質量	&mass	&$M, m$	&kg	&kilogram	& p.~44, 59 \\ \hline
  ばねを単位長さだけ伸縮させる力	&ばね定数	&spring constant	&$k$	&N/m	&newton per meter	& p.~47 \\ \hline
  接触面から垂直に働く力	&垂直抗力	&normal reaction	&$N$	&N	&newton	& p.~49 \\ \hline
  接触面で水平に働く抵抗力	&摩擦力	&friction	&$f$	&N	&newton	& p.~49 \\ \hline
  摩擦力と垂直抗力との比	&摩擦係数	&coefficient of friction	&$\mu$	&---	&---	& p.~51 \\ \hline
  接続点から糸の向きに働く力	&張力	&tension	&$T$	&N	&newton	& p.~65 \\ \hline\hline
  力と時間との積	&力積	&impulse	&$I$	&N$\cdot$s	&newton second	& p.~77 \\ \hline
  移動する運動の勢い(速さと質量の積)	&運動量	&momentum	&$p$	&kg$\cdot$m/s	&kilogram meter per second	& p.~78 \\ \hline
  跳ね返りやすさ	&反発係数	&coefficent of restitution	&$e$	& ---	& ---	& p.~8, 91\\ \hline\hline
  力とその力の向きに動いた距離との積	&仕事	&work	&$W$	&J	&joule	& p.~94 \\ \hline
  単位時間あたりにする仕事	&仕事率	&power	&$P$	&W	&watt	& p.~98 \\ \hline
  仕事をすることができる能力	&エネルギー	&energy	&$E, U, K$	&J	&joule	& p.~100 \\ \hline
  運動している物体が持つエネルギー	&運動エネルギー	& kinetic energy	&$K$	&J	&joule	& p.~100 \\ \hline
  位置の変化に伴うエネルギー	&位置エネルギー	& potential energy	&$U$	&J	&joule	& p.~104 \\ \hline
\end{tabular}

*英語の綴りは、電子辞書で発音も確認してください.
*量記号は斜体(イタリック体)で書きます.

*単位記号は立体(ローマン体)で書きます.

*人命由来の単位記号は大文字で書き始めます.しかし、綴りの時には小文字で書きます.



\end{document}
